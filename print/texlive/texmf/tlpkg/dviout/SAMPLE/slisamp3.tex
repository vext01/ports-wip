\documentclass{slides}
\usepackage{graphicx, color, myhyper}
%\def\pause{\special{pause}}
\def\pause{}

\paperwidth=10.24in
\paperheight=7.68in
\textwidth=9in
\textheight=6.6in
\voffset=-.9in
\hoffset=-.48in

\title{Presentation with dviout%
\special{dviout `keep cmode}%
%  color(def), Draw Off, cover sheet Off, general screen, Presentation ;
%  Screen: 1024 x 768, Reverse ;
%  Draw On, Fit
\special{dviout -cmode=2 !ANFN0!p; -e=0 -y=XGAP !v;!bdf}%
\pagecolor{blue}%
}
\author{Toshio OSHIMA}
\date{January 1, 2000}
\pagestyle{plain}

%%%%%%    TEXT START    %%%%%%
\begin{document}
\maketitle

\begin{slide}
\special{dviout `keep cmode}%%	for non Presentation Mode
\special{dviout -cmode=2 !G}%%	for non Presentation Mode + ...
\pagecolor{blue}%
{\large \S0 Presentation by \TeX}

Many technical documents including mathematical formulas are written by
{\TeX} and they can be used for presentations by dviout for Windows.

{\color{red} Use key ([Space] etc.) or mouse to scroll the screen!}

\pause
Study the solution $u(x)$ of the Shr\"odinger equation
\[
 \left(-\frac12\sum_{1\le j\le n}\frac{\partial^2}{\partial x_j^2} + U(x)\right)u(x) = \lambda u(x)
\]

with the potential function

\pause
\[
 U(x) = \sum_{1\le i < j\le n}\frac{C}{\sinh^2(x_i - x_j)}.
\]

\pause
It can be written by Gauss hypergeometric function when $n=2$.
\end{slide}
%%%%%%%%%%%%%%%%%%%%%%%%%%%%%%%
\begin{slide}
\special{dviout !G}%
\pagecolor[rgb]{0,0.3,0}%% 	to see the hot spot in non-revese mode
One may try to present this document by a key or mouse operation.
If it is displayed under a cover sheet,
\begin{itemize}
\item
push [Space] key to proceed step by step,
\item
move the mouse under pushing its left button.
\end{itemize}
For this purpose, open \href{file:slisamp2.dvi}{slisamp2.dvi} (a sample 
with the slightly yellow background).

Or open \href{file:slisamp3.dvi}{slisamp3.dvi} or \href{file:slisamp4.dvi}{slisamp4.dvi} (without the cover sheet).

Otherwise jump to \href{file:sample.dvi}{sample.dvi}. % after 30 seconds. 

Note that the [ESC] key is the toggle switch between the presentation mode and
the normal mode.
\end{slide}
\end{document}
